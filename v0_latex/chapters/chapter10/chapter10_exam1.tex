\section{Ethereum 101 Quiz Solutions}

\textbf{1. Which of the following EVM components is/are non-volatile across transactions?}\label{sec:exam1_q1}\\

\textbf{Solution: C.}

A stack machine is a computer processor or a virtual machine in which the primary interaction is moving short-lived temporary values to and from a push down stack.\\

The EVM is a simple stack-based architecture consisting of the stack, volatile memory, non-volatile storage with a word size of 256-bit (chosen to facilitate the Keccak256 hash scheme and elliptic-curve computations) and Calldata.\\

Calldata is a read-only byte-addressable space where the data parameter of a transaction or call is held.\\

\textbf{2. The number of transactions in a Ethereum block depends on}\label{sec:exam1_q2}\\

\textbf{Solution: B \& C.}

Block gas limit is set by miners and refers to the cap on the total amount of gas expended by all transactions in the block, which ensures that blocks can't be arbitrarily large.
Blocks therefore are not a fixed size in terms of the number of transactions because different transactions consume different amounts of gas.\\

\textbf{3. EVM Stores}\label{sec:exam1_q3}\\

\textbf{Solution: A \& C.}

EVM uses big-endian ordering where the most significant byte of a word is stored at the smallest memory address and the least significant byte at the largest\\

\textbf{4. Ethereum's threat model is characterised by}\label{sec:exam1_q4}\\

\textbf{Solution: D.}

Given the aspirational absence of trusted intermediaries, everyone and everything is meant to be untrusted by default.
Participants in this model include developers, miners/validators, infrastructure providers and users, all of whom could potentially be adversaries.\\

\textbf{5. Ethereum smart contracts do not run into halting problems because}\label{sec:exam1_q5}\\

\textbf{Solution: C.}

Turing-complete systems face the challenge of the halting problem i.e. given an arbitrary program and its input, it is not solvable to determine whether the program will eventually stop running.
So Ethereum cannot predict if a smart contract will terminate, or how long it will run.
Therefore, to constrain the resources used by a smart contract, Ethereum introduces a metering mechanism called gas.\\

\textbf{6. Which of the following operation(s) touch(es) storage?}\label{sec:exam1_q6}\\

\textbf{Solution: B.}

Most EVM instructions operate with the stack (stack-based architecture) and there are also stack-specific operations e.g. \verb|PUSH|, \verb|POP|, \verb|SWAP|, \verb|DUP| etc\dots
Storage is a 256-bit to 256-bit key-value store.
This is accessed with \verb|SLOAD|/\verb|SSTORE| instructions.\\

\textbf{7. The most gas-expensive operation is}\label{sec:exam1_q7}\\

\textbf{Solution: C.}

\verb|SLOAD| is 2100 gas and \verb|SSTORE| is 20000 gas to set a storage slot from 0 to non-0 and 5000 gas otherwise.
\verb|CREATE| is 32000 gas and \verb|SELFDESTRUCT| is 5000 gas.\\

\pagebreak

\textbf{8. Transaction \texttt{T1} attempts to write to storage values \texttt{S1} and \texttt{S2} of contract \texttt{C}.
Transaction \texttt{T2} attempts to read the same storage values \texttt{S1} and \texttt{S2}.
However, \texttt{T1} reverts due to an exception after writing \texttt{S2}. Which or the following is/are TRUE?}\label{sec:exam1_q8}\\

\textbf{Solution: B.}

Of the transaction properties: atomicity (it is all or nothing i.e. cannot be divided or interrupted by other transactions).\\

A transaction reverts for different exceptional conditions such as running out of gas, invalid instructions etc\dots In which case all state changes made so far are discarded and the original state of the account is restored as it was before this transaction executed.\\

\textbf{9. The gas tracking website \url{https://etherscan.io/gastracker} says that Low gas cost is 40 gwei.
This affects}\label{sec:exam1_q9}\\

\textbf{Solution: A.}

The \verb|gasPrice| is the price a transaction originator is willing to pay in exchange for gas.
The price is measured in wei per gas unit.
The higher the gas price, the faster the transaction is likely to be confirmed on the blockchain.
The suggested gas price depends on the demand for block space at the time of the transaction.\\

In addition, \verb|gasLimit| is the maximum amount of gas the originator is willing to pay for this transaction value: the amount of Ether (in wei) to send to the destination.\\

\textbf{10. Security of Ethereum DApps depends on}\label{sec:exam1_q10}\\

\textbf{Solution: A, B \& C.}

On-chain vs Off-chain: Smart contracts are ``on-chain" Web3 components and they interact with ``off-chain" components that are very similar to Web2 software.
So the major differences in security perspectives between Web3 and Web2 mostly narrow down to security considerations of smart contracts vis-a-vis Web2 software.\\

\textbf{11. A \texttt{nonce} is present in}\label{sec:exam1_q11}\\

\textbf{Solution: C.}

One of the fields in an Ethereum account is the \verb|nonce|, which is a counter used to make sure each transaction can only be processed once.\\

Furthermore, a transaction is a serialized binary message that also contains a \verb|nonce|, which is again a sequence number, issued by the originating EOA, used to prevent message replay.\\

\textbf{12. Miners are responsible for setting}\label{sec:exam1_q12}\\

\textbf{Solution: B.}

Gas price: The price a transaction originator is willing to pay in exchange for gas.
The price is measured in wei per gas unit.
The higher the gas price, the faster the transaction is likely to be confirmed on the blockchain.
The suggested gas price depends on the demand for block space at the time of the transaction.\\

Block gas limit is set by miners and refers to the cap on the total amount of gas expended by all transactions in the block, which ensures that blocks can't be arbitrarily large.\\

\textbf{13. Which of the following information CANNOT be obtained in the EVM?}\label{sec:exam1_q13}\\

\textbf{Solution: Correct is B \& D.}

Recall the following opcode: \verb|0x40 BLOCKHASH 1 1|.
It gets the hash of one of the 256 most recent complete blocks (so, not any block).
In addition, there's no operation to access transaction logs.\\

\textbf{14. Miners are incentivized to validate and create new blocks by}\label{sec:exam1_q14}\\

\textbf{Solution: A \& C.}

Miners are rewarded for blocks accepted into the blockchain with a block reward in ether (currently 2 ETH).
A miner also gets fees which is the ether spent on gas by all the transactions included in the block.\\

\textbf{15. Smart contracts on Ethereum}\label{sec:exam1_q15}\\

\textbf{Solution: A \& C.}

Web3 is a permissionless, trust-minimized and censorship-resistant network for transfer of value and information.\\

\textbf{16. Hardfork on Ethereum}\label{sec:exam1_q16}\\

\textbf{Solution: C.}

A hard fork is planned soon, to introduce an exponential difficulty increase, and to motivate a transition to PoS when ready.\\

\pagebreak

\textbf{17. Which call instruction could be used to allow modifying the caller account’s state?}\label{sec:exam1_q17}\\

\textbf{Solution: B \& C.}

Recall the fllowing opcodes
\begin{lstlisting}[style=defaultStyle]
0xf1 CALL 7 1 Message-call into an account
0xf2 CALLCODE 7 1 Message-call into this account with an alternative account's code
0xf4 DELEGATECALL 6 1 Message-call into this account with an alternative account's code, but persisting the current values for sender and value
0xfa STATICCALL 6 1 Static message-call into an account
\end{lstlisting}

Another variant of \verb|CALL| is \verb|DELEGATECALL|, which essentially runs the code of another contract inside the context of the execution of the current contract.\\

\textbf{18. The length of addresses on Ethereum is}\label{sec:exam1_q18}\\

\textbf{Solution: B.}

Ethereum state is made up of objects called ``accounts", with each account having a 20-byte address and state transitions being direct transfers of value and information between accounts.\\

\textbf{19. Which of the following statements is/are TRUE about gas?}\label{sec:exam1_q19}\\

\textbf{Solution: B.}

Any unused gas in a transaction (\verb|gasLimit| minus gas used by the transaction) is refunded to the sender's account at the same \verb|gasPrice|.
Ether used to purchase gas used for the transaction is credited to the beneficiary address (specified in the block header), the address of an account typically under the control of the miner.
These are the transaction ``fees" paid to the miner.\\

\textbf{20. The high-level languages typically used for writing Ethereum smart contracts are}\label{sec:exam1_q20}\\

\textbf{Solution: C \& D.}

\verb|Solidity| language continues to dominate smart contracts without much real competition (except \verb|Viper| perhaps).\\

\textbf{21. Ethereum nodes talk to each other via}\label{sec:exam1_q21}\\

\textbf{Solution: A.}

Ethereum node/client: A node is a software application that implements the Ethereum specification and communicates over the peer-to-peer network with other Ethereum nodes.\\

\pagebreak

\textbf{22. EVM is not a von Neumann architecture because}\label{sec:exam1_q22}\\

\textbf{Solution: B.}

EVM does not follow the standard von Neumann architecture.
Rather than storing program code in generally accessible memory or storage, it is stored separately in a virtual ROM accessible only through a specialized instruction.\\

In computer science, a universal Turing machine (UTM) is a Turing machine that simulates an arbitrary Turing machine on arbitrary input.
This principle is considered to be the origin of the idea of a stored-program computer used by John von Neumann in 1946 for the "Electronic Computing Instrument" that now bears von Neumann's name: the von Neumann architecture.\\

\textbf{23. User \texttt{A} sends transaction \texttt{T1} from address \texttt{A1} with \texttt{gasPrice} \texttt{G1} and later transaction \texttt{T2} from address \texttt{A2} with \texttt{gasPrice} \texttt{G2}}\label{sec:exam1_q23}\\

\textbf{Solution: D.}

Transaction inclusion is not guaranteed and depends on network congestion and \verb|gasPrice| among other things.
Miners determine inclusion.\\

Transaction order is not guaranteed either, and depends on network congestion and \verb|gasPrice| among other things.
Miners determine order as well.\\

\textbf{24. The types of accounts on Ethereum are}\label{sec:exam1_q24}\\

\textbf{Solution: C.}

Ethereum has two different types of accounts: externally Owned Accounts (EOAs) controlled by private keys, and contract Accounts controlled by their contract code.\\

\textbf{25. The number of decimals in Ether is}\label{sec:exam1_q25}\\

\textbf{Solution: C.}

Ether is subdivided into smaller units, being the smallest unit named wei.
$10^{18}$ wei is 1 Ether.\\

\textbf{26. The difference(s) between Bitcoin and Ethereum is/are}\label{sec:exam1_q26}\\

\textbf{Solution: A, B \& C.}

Ethereum uses Bitcoin's consensus model: Nakamoto Consensus.\\

\pagebreak

\textbf{27. Security Audits for smart contracts are performed because}\label{sec:exam1_q27}\\

\textbf{Solution: C.}

Secure Software Development Lifecycle (SSDLC) processes for Web2 products have evolved over several decades to a point where they are expected to meet some minimum requirements of a combination of internal validation, external assessments (e.g. product/process audits, penetration testing) and certifications depending on the value of managed assets, anticipated risk, threat model and the market domain of products (e.g. financial sector has stricter regulatory compliance requirements).\\

\textbf{28. The number of modified Merkle-Patricia trees in an Ethereum block is}\label{sec:exam1_q28}\\

\textbf{Solution: C.}

Blocks contain block header, transactions and ommers' block headers.
Block header contains \verb|stateRoot|, \verb|transactionsRoot| and \verb|receiptsRoot| are 256-bit hashes of the root nodes of modified Merkle-Patricia trees.\\

\textbf{29. EVM opcodes}\label{sec:exam1_q29}\\

\textbf{Solution: B \& D.}

The code in Ethereum contracts is written in a low-level, stack-based bytecode language, referred to as "Ethereum virtual machine code" or "EVM code".
The code consists of a series of bytes (hence called bytecode), where each byte represents an operation.\\

Most EVM instructions operate with the stack (stack-based architecture) and there are also stack-specific operations e.g. \verb|PUSH|, \verb|POP|, \verb|SWAP|, \verb|DUP| etc.\\

\textbf{30. Gas for EVM opcodes}\label{sec:exam1_q30}\\

\textbf{Solution: B.}

Gas costs for different instructions are different depending on their computational/storage load on the client.\\

A hard fork to change the gas calculation for certain I/O-heavy operations and to clear the accumulated state from a denial-of-service (DoS) attack that exploited the low gas cost of those operations.\\

A hard fork to address more DoS attack vectors, and another state clearing.
Also, a replay attack protection mechanism.\\

\pagebreak

\textbf{31. Ethereum Virtual Machine is a}\label{sec:exam1_q31}\\

\textbf{Solution: B.}

The EVM is a simple stack-based architecture consisting of the stack, volatile memory, non-volatile storage with a word size of 256-bit (chosen to facilitate the Keccak256 hash scheme and elliptic-curve computations) and Calldata.\\

\textbf{32. Which of the following statement(s) is/are FALSE?}\label{sec:exam1_q32}\\

\textbf{Solution: B, C \& D.}

\begin{lstlisting}[style=defaultStyle]
0x31 BALANCE 1 1 Get balance of the given account
0x3f EXTCODEHASH 1 1 Get hash of an account's code
0x40 BLOCKHASH 1 1 Get the hash of one of the 256 most recent complete blocks
0x43 NUMBER 0 1 Get the block's number
\end{lstlisting}