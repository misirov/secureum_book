\section{Audit Techniques \& Tools 101 Quiz}

\textbf{1. Which of the below is/are accurate?}

\begin{enumerate}[label=\Alph*.]
    \item Audits identify all security vulnerabilities and guarantee bug-free code.
    \item Audits cover only smart contracts but never the offchain code.
    \item Audits suggest fixes for issues identified and aim to reduce risk.
    \item None of the above.
\end{enumerate}

Find the solution \hyperref[sec:exam6_q1]{here}.\\

\textbf{2. Audit reports from audit firms tipically include}

\begin{enumerate}[label=\Alph*.]
    \item Finding likelihood/difficulty, impact and severity.
    \item Exploit scenarios and recommended fixes.
    \item Formal verification of all findings with proofs and counterexamples.
    \item All of the above.
\end{enumerate}

Find the solution \hyperref[sec:exam6_q2]{here}.\\

\textbf{3. These audit techniques are especially well-suited for smart contracts (compared to Web2 programs)}

\begin{enumerate}[label=\Alph*.]
    \item Formal verification because contracts are relatively smaller with specific properties.
    \item Fuzzing because anyone can send random inputs to contracts on blockchain.
    \item Static source-code analysis because contracts are expected to be open source.
    \item High-coverage testing because contract states and transitions are relatively fewer.
\end{enumerate}

Find the solution \hyperref[sec:exam6_q3]{here}.\\

\pagebreak

\textbf{4. The following kinds of findings may be expected during audits}

\begin{enumerate}[label=\Alph*.]
    \item True positives after confirmation from the project team.
    \item False positives due to assumptions from missing specification and threat model.
    \item False positives due to limitations of time and expertise.
    \item None of the above.
\end{enumerate}

Find the solution \hyperref[sec:exam6_q4]{here}.\\

\textbf{5. Which of the following is/are true?}

\begin{enumerate}[label=\Alph*.]
    \item Audited projects always have clear/complete specification and documentation of all contract properties.
    \item Manual analysis is typically required for detecting application logic vulnerabilities.
    \item Automated tools like Slither and MythX have no false negatives.
    \item The project team always fixes all the findings identified in audits.
\end{enumerate}

Find the solution \hyperref[sec:exam6_q5]{here}.\\

\textbf{6. Automated tools for smart contract analysis}

\begin{enumerate}[label=\Alph*.]
    \item Are sufficient therefore making manual analysis unnecessary.
    \item Have no false positives whatsoever.
    \item Are best-suited for application-level vulnerabilities.
    \item None of the above.
\end{enumerate}

Find the solution \hyperref[sec:exam6_q6]{here}.\\

\textbf{7. Which of the following is/are true?}

\begin{enumerate}[label=\Alph*.]
    \item Slither supports detectors, printers, tools and custom analyses.
    \item Echidna is a symbolic analyzer tool.
    \item MythX is a combination of static analysis, symbolic checking and fuzzing tools.
    \item None of the above.
\end{enumerate}

Find the solution \hyperref[sec:exam6_q7]{here}.\\

\pagebreak

\textbf{8. Which of the following is/are correct about false positives?}

\begin{enumerate}[label=\Alph*.]
    \item They are findings that are not real concerns/vulnerabilities after further review.
    \item They are real vulnerabilities but are falsely claimed by auditors as benign.
    \item They are possible with automated tools.
    \item None of the above.
\end{enumerate}

Find the solution \hyperref[sec:exam6_q8]{here}.\\

\textbf{9. Audit findings}

\begin{enumerate}[label=\Alph*.]
    \item May include both specific vulnerabilities and generic recommendations.
    \item May not all be fixed by the project team for reasons of relevancy and acceptable trust/threat model.
    \item Always have demonstrable proof-of-concept excploit code on mainnet.
    \item None of the above.
\end{enumerate}

Find the solution \hyperref[sec:exam6_q9]{here}.\\

\textbf{10. Which of the following is/are typicial manual review approach(es)?}

\begin{enumerate}[label=\Alph*.]
    \item Asset flow.
    \item Symbolic checking.
    \item Inferring constraints.
    \item Evaluating assumptions.
\end{enumerate}

Find the solution \hyperref[sec:exam6_q10]{here}.\\

\textbf{11. Access control analysis is a crytical part of manual review for the reason(s) that}

\begin{enumerate}[label=\Alph*.]
    \item It is the easiest to perform because smart contracts never have access control
    \item It is the fastest to perform because there are always only two roles: users and admins.
    \item It is fundamental to security because privileged roles (of which there may be many) may be misused/compromised.
    \item None of the above.
\end{enumerate}

Find the solution \hyperref[sec:exam6_q11]{here}.\\

\pagebreak

\textbf{12. Which of the following is/are true about vulnerability difficulty and impact?}

\begin{enumerate}[label=\Alph*.]
    \item Difficulty indicates how hard it was for auditors to detect the issue.
    \item Difficulty is an objective measure that can always be quantified.
    \item Impact is tipically classified as High if there is loss/lock of funds.
    \item None of the above.
\end{enumerate}

Find the solution \hyperref[sec:exam6_q12]{here}.\\

\textbf{13. Application-level security constraints}

\begin{enumerate}[label=\Alph*.]
    \item Are always clearly/completely specified and documented.
    \item Have to be typically inferred from the code or discussions with project team.
    \item Typically require manual analysis.
    \item None of the above.
\end{enumerate}

Find the solution \hyperref[sec:exam6_q13]{here}.\\

\textbf{14. Which of the following is/are typically true?}

\begin{enumerate}[label=\Alph*.]
    \item Static analysis analyzes program properties by actually executing the program.
    \item Fuzzing uses valid, expected and deterministic inputs.
    \item Symbolic checking enumerates individual states/transitions or efficient state space traversal.
    \item None of the above.
\end{enumerate}

Find the solution \hyperref[sec:exam6_q14]{here}.\\

\textbf{15. Which of the following is/are generally true about asset flow analysis?}

\begin{enumerate}[label=\Alph*.]
    \item Analyzes the flow of Ether or tokens managed by smart contracts.
    \item Assets should be withdrawn only by authorized addresses.
    \item The timing aspects of asset withdrawals/deposits is irrelevant.
    \item The type and quantity of asset withdrawals/deposits is irrelevant.
\end{enumerate}

Find the solution \hyperref[sec:exam6_q15]{here}.\\

\textbf{16. Which of the following is/are generally true about control and data flow analyses?}

\begin{enumerate}[label=\Alph*.]
    \item Interprocedural control flow is typically indicated by a call graph.
    \item Intraprocedural control flow is dictated by conditionals (if/else), loops (for/while/do/continue/break) and return statements.
    \item Interprocedural data flow is evaluated by analyzing the data used as argument values for function parameters at call sites.
    \item Intraprocedural data flow is evaluated by analyzing the assignment and use of variables/constants along control flow paths within functions
\end{enumerate}

Find the solution \hyperref[sec:exam6_q16]{here}.\\
