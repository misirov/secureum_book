\section{RACE 4 Quiz Solutions}

\textbf{1. \texttt{InSecureum} implements}\label{sec:race4_q1}\\

\textbf{Solution: A \& B.}

The decimals value follows the standard but it typically returns \verb|18| (\verb|8| is atypical), imitating the relationship between Ether and Wei.\\

The \verb|decreaseAllowance| and \verb|increaseAllowance| functions were introduced in the \verb|OpenZeppelin ERC20| implementation to mitigate frontrunning issues of the standard approve, but they are not part of the \verb|ERC20| standard.\\

The \verb|transfer| function is part of the standard though.\\

\textbf{2. In \texttt{InSecureum}}\label{sec:race4_q2}\\

\textbf{Solution: A \& C.}

Since \verb|decimals()| returns a constant hardcoded value without accessing storage other non-calldata information it can indeed be declared as \verb|pure|.\\

Generally, functions prefixed with underscores should be internal or should not have the prefix.
Making \verb|_burn()| external would currently allow anyone to burn anyone else's balance.
And the fact that \verb|_mint()| is currently external allows anyone to mint as many InSecureum tokens as they wish.\\

\textbf{3. \texttt{InSecureum transferFrom()}}\label{sec:race4_q3}\\

\textbf{Solution: A, B \& C.}

The subtraction within the \verb|unchecked| block effectively allows anyone to steal anyone else's full token balance since subtracting from an allowance of 0 will cause an integer underflow and the allowance value will wrap since \linebreak\verb|0 - 1 == type(uint256).max|).\\

The fact that the function won't revert when subtracting from the allowance due to the \verb|unchecked| block, can by itself be seen as an incorrect allowance check.
The other check, skipping allowance subtraction when an ``infinite approval'' was given by setting the allowance to the maximum value of \verb|uint256|, appears to be correct.
Since the special handling for unlimited approvals prevents unnecessary storage updates, it is indicative of an optimization.\\

\pagebreak

\textbf{4. In \texttt{InSecureum}}\label{sec:race4_q4}\\

\textbf{Solution: C \& D.}

Neither function make use of the \verb|unchecked| block which would allow integer overflows to happen in this version of \verb|Solidity|.\\

The \verb|decreaseAllowance| function does indeed not allow reducing the allowance to zero since the requirement enforces that the \verb|subtractedValue| must always be smaller than \verb|currentAllowance|.
It would be better to use \verb|>=| here to allow allowance reductions to zero.
That requirement does make \verb|Solidity|'s own integer underflow check for \verb|currentAllowance - subtractedValue| redundant, so it could indeed be optimised with \verb|unchecked{}|.\\

\textbf{5. \texttt{InSecureum \_transfer()}}\label{sec:race4_q5}\\

\textbf{Solution: D.}

All of the addresses \verb|_transfer()| uses are checked to make sure they're not zero-addresses.
Neither integer overflows nor underflows are possible with this \verb|Solidity| version without the use of \verb|unchecked{}|.\\

\textbf{6. \texttt{InSecureum \_mint()}}\label{sec:race4_q6}\\

\textbf{Solution: A \& C.}

The \verb|_mint| function is currently not ensuring that the receiving address is non-zero.
The event emission appears to be correctly following \verb|IERC20|:

\begin{lstlisting}[language=Solidity, style=solStyle]
event Transfer(address indexed from, address indexed to, uint256 value);
\end{lstlisting}

This mint implementation overwrites the account's current balance instead of adding to it.\\

\textbf{7. \texttt{InSecureum \_burn()}}\label{sec:race4_q7}\\

\textbf{Solution: B.}

It correctly applies zero-address validation on the account to burn from.
The event permission is incorrect, \verb|from| and \verb|to| need to be switched around to follow the \verb|IERC20| interface:

\begin{lstlisting}[language=Solidity, style=solStyle]
event Transfer(address indexed from, address indexed to, uint256 value);
\end{lstlisting}

The balance update is correct and although an \verb|unchecked| block is used, no underflow can happen thanks to the requirement before.\\

\pagebreak

\textbf{8. \texttt{InSecureum \_approve()}}\label{sec:race4_q8}\\

\textbf{Solution: B \& C.}

Although no zero-address validation is missing the error messages have been confused with each other.
The update of allowances is currently incorrect since it only adds the amount to the current allowance instead of setting it to the amount overwriting the old value.\\