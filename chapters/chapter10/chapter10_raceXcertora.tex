\section{RACE-X: Certora Quiz Solutions}

\textbf{1. When is the expression $p\Rightarrow q$ false?}\label{sec:raceXcertora_q1}\\

\textbf{Solution: D.}

\textbf{2. Is the following expression true? $p\wedge q\Rightarrow p$}\label{sec:raceXcertora_q2}\\

\textbf{Solution: C.}

\textbf{3. Is the following expression true? $(p\wedge(q\vee\neg p))\wedge\neg q$}\label{sec:raceXcertora_q3}\\

\textbf{Solution: B.}

\textbf{4. Is the following expression true? $\neg (\neg p\vee\neg q\vee\neg r)\Leftrightarrow p\wedge q\wedge r$}\label{sec:raceXcertora_q4}\\

\textbf{Solution: A.}

\textbf{5. Is the following expression true? $\neg((\neg p\vee\neg q)\vee (\neg p\wedge\neg q))$}\label{sec:raceXcertora_q5}\\

\textbf{Solution: C.}

\textbf{6. Is the following expresison true? $\neg((p\wedge q)\vee(\neg p\wedge\neg q))$}\label{sec:raceXcertora_q6}\\

\textbf{Solution: C.}

\textbf{7. Given the expression: $M:p\wedge\vee\neg p$\\
Which of the following expressions implies the above given expression $M$?}\label{sec:raceXcertora_q7}\\

\textbf{Solution: A, B, C \& D.}

\pagebreak

\textbf{8. Given the expression: $M:p\wedge\wedge\neg p$\\
Which of the following expressions implies the above given expression $M$?}\label{sec:raceXcertora_q8}\\

\textbf{Solution: A, B, C \& D.}

\textbf{9. Is the following expression true? $(p\Rightarrow (q\Rightarrow r))\Rightarrow((p\Rightarrow q)\Rightarrow r)$}\label{sec:raceXcertora_q9}\\

\textbf{Solution: C.}

\textbf{10. Is the following expression true? $(()p\Rightarrow q)\Rightarrow r)\Rightarrow(p\Rightarrow (q\Rightarrow r))$}\label{sec:raceXcertora_q10}\\

\textbf{Solution: A.}

\textbf{11. Given the below four properties for \href{https://github.com/OpenZeppelin/openzeppelin-contracts/blob/master/contracts/token/ERC20/ERC20.sol}{OpenZeppelin's ERC20 implementation}, answer questions 11 to 13}\\

\textbf{P1}: Forall user: \verb|balanceOf(user)| can only change on \verb|mint()|, \verb|burn()|, \verb|transfer()|, \verb|transferFrom()|\\
\textbf{P2}: \verb|totalSupply()| is the sum of \verb|balanceOf()| over all users\\
\textbf{P3}: Forall user: \verb|balanceOf(user) <= totalSupply()|\\
\textbf{P4}: Forall user: \verb|balanceOf(user)| can only change on operation performed when \verb|msg.sender == user| or when \verb|allowance(user, msg.sender)| is not zero\\

\textbf{Which of the below properties are correct properties of \texttt{ERC20}?}\label{sec:raceXcertora_q11}\\

\textbf{Solution: A, B \& C.}

\textbf{12. Given the four properties for \href{https://github.com/OpenZeppelin/openzeppelin-contracts/blob/master/contracts/token/ERC20/ERC20.sol}{OpenZeppelin's ERC20 implementation} shown in question 11,\\ and given the following buggy version of transferFrom:}

\begin{lstlisting}[language=Solidity, style=solStyle]
function transferFrom(address _from, address _to, uint256 _value) public returns (bool) {
   require(_to != address(0));
   require(_value <= balances[_from]);
   require(_value <= allowed[_from][msg.sender]);
   balances[_from] = balances[_from].add(_value);
   balances[_to] = balances[_to].sub(_value);
   allowed[_from][msg.sender] = allowed[_from][msg.sender].sub(_value);
   emit Transfer(_from, _to, _value);
   return true;
 }
\end{lstlisting}

\textbf{Which of the below properties are violated?}\label{sec:raceXcertora_q12}\\

\textbf{Solution: D.}

\textbf{13. Given the four properties for \href{https://github.com/OpenZeppelin/openzeppelin-contracts/blob/master/contracts/token/ERC20/ERC20.sol}{OpenZeppelin's ERC20 implementation} shown in question 11,\\ Is there an implication between P3 and P2?}\label{sec:raceXcertora_q13}\\

\textbf{Solution: Correct is A.}

\textbf{14. Assuming a correct implementation of \texttt{transferFrom}, as in \href{https://github.com/OpenZeppelin/openzeppelin-contracts/blob/master/contracts/token/ERC20/ERC20.sol}{OpenZeppelin's ERC20 implementation}, and given the following pseudo unit-test code}

\begin{lstlisting}[language=Solidity, style=Solstyle]
uint256  bFrom = balances[from];
uint256 bTo = balances[to];
transferFrom(from, to, x);
assert($exp$);
\end{lstlisting}

\textbf{Which of the following expressions are valid choices (should always hold) for \texttt{\$exp\$}?}\label{sec:raceXcertora_q14}\\

\textbf{Solution: B \& C.}

\textbf{15. The below contract (same contract for questions 15 to 17) has a bug:}

\begin{lstlisting}[language=Solidity, style=solStyle]
contract test {
   // Assume other required functionality is correctly implemented
   uint256 private constant MAX_FUND_RAISE = 100 ether;
   mapping (address => uint256) contributions;

    function contribute() external payable {
       require(address(this).balance != MAX_FUND_RAISE);
       contributions[msg.sender] += msg.value;
   }
}
\end{lstlisting}

\textbf{Which of the following invariants should hold on a correct implementation of the contribute function?}\\

\textbf{P1}: Forall user: \verb|address(this).balance <= contributions[user]|\\
\textbf{P2}: Forall user: \verb|contributions[user] <= address(this).balance|\\
\textbf{P3}: Forall user: \verb|contributions[user] <= MAX_FUND_RAISE|\\
\textbf{P4}: \verb|address(this).balance <= MAX_FUND_RAISE|\label{sec:raceXcertora_q15}\\

\textbf{Solution: B, C \& D.}

\textbf{16. Following the same contract of question 15, which of the following invariant(s) that is/are supposed to hold is/are violated due to the buggy implementation?}

\textbf{P1}: Forall user: \verb|address(this).balance <= contributions[user]|\\
\textbf{P2}: Forall user: \verb|contributions[user] <= address(this).balance|\\
\textbf{P3}: Forall user: \verb|contributions[user] <= MAX_FUND_RAISE|\\
\textbf{P4}: \verb|address(this).balance <= MAX_FUND_RAISE|\label{sec:raceXcertora_q16}\\

\textbf{Solution: C \& D.}

\pagebreak

\textbf{17. Following the same contract of question 15, the revert characteristic (conditions in which the function should revert) of a correct implementation of \texttt{contribute} is}\label{sec:raceXcertora_q17}\\

\textbf{Solution: B.}

\textbf{18. In the below contract (same contract for questions 18 \& 19)}

\begin{lstlisting}[language=Solidity, style=solStyle]
pragma solidity 0.7.0;

contract InSecureumToken {

mapping(address => uint) private balances;
uint public decimals = 10**18; // decimals of the token
uint public totalSupply; // total supply
uint MAX_SUPPLY = 100 ether; // Maximum total supply
event Mint(address indexed destination, uint amount);

function balanceOf(address u) public returns (uint256) {
   return balances[u];
}

function ethBalance(address u) public returns (uint256) {
   return u.balance;
}

function transfer(address to, uint amount) public {
   // save the balance in local variables
   // so that we can re-use them multiple times
   // without paying for SLOAD on every access
   uint balance_from = balances[msg.sender];
   uint balance_to = balances[to];
   require(balance_from >= amount);
   balances[msg.sender] = balance_from - amount;
   balances[to] = safeAdd(balance_to, amount);
}

/// @notice Allow users to buy a token. 1 ether = 10 tokens
/// @dev Users can send more ether than token to be bought, to donate a fee to the protocol team.
function buy(uint desired_tokens) public payable {
    // Check if enough ether has been sent
    uint required_wei_sent = (desired_tokens / 10) * decimals;
    require(msg.value >= required_wei_sent);
    // Mint the tokens
    totalSupply = safeAdd(totalSupply, desired_tokens);
    balances[msg.sender] = safeAdd(balances[msg.sender], desired_tokens);
    emit Mint(msg.sender, desired_tokens);
}

/// @notice Add two values. Revert if overflow
function safeAdd(uint a, uint b) pure internal returns(uint) {
    if (a + b < a) {
        revert();
    }
    return a + b;
}
}
\end{lstlisting}

\textbf{Given the following two properties:}
\textbf{P1}: \verb|totalSupply()| is the sum of \verb|balanceOf()| over all users.\\
\textbf{P2}: Monotonicity of \verb|totalSupply| vs the contract's ether balance:
\begin{enumerate}[label=(\alph*)]
    \item\verb|totalSupply| is increased iff ($\Leftrightarrow$) \verb|this.balance| is increased and
    \item\verb|totalSupply| is decreased iff ($\Leftrightarrow$) \verb|this.balance| is decreased
\end{enumerate}

\textbf{Which of the existing issues in the code violates which property?}\label{sec:raceXcertora_q18}\\

\textbf{Solution: B \& C.}

Ethereum state is made up of objects called ``accounts", with each account having a 20-byte address and state transitions being direct transfers of value and information between accounts.\\

\textbf{19. Following the same contract of question 18 and assuming a correct implementation of \texttt{buy()} and \texttt{transfer()}, which properties should hold?}\label{sec:raceXcertora_q19}\\

\textbf{Solution: B.}

\textbf{20. In \href{https://github.com/OpenZeppelin/openzeppelin-contracts/blob/master/contracts/token/ERC721/ERC721.sol}{OpenZeppelin’s implementation of ERC721}, which of the following properties are correct specification assuming \texttt{user != 0}?}\label{sec:raceXcertora_q20}\\

\textbf{Solution: C.}

\textbf{21. In \href{https://github.com/OpenZeppelin/openzeppelin-contracts/blob/master/contracts/token/ERC721/ERC721.sol}{OpenZeppelin’s implementation of ERC721}, which of the following is necessarily correct after a successful (non-reverting) call to \texttt{transferFrom(from, to, tokenId)}?}\label{sec:raceXcertora_q21}\\

\textbf{Solution: A \& C.}

\pagebreak

\textbf{22. In \href{https://github.com/OpenZeppelin/openzeppelin-contracts/blob/master/contracts/token/ERC721/extensions/ERC721Enumerable.sol}{OpenZeppelin’s implementation of ERC721 Enumerable}, which of the following expressions is true?}\label{sec:raceXcertora_q22}\\

\textbf{Solution: A \& C.}

\textbf{23. Based on the lecture on \href{https://www.youtube.com/watch?v=VGSsPIsbb6U}{"Auditing and Formal Verification: Better together"} by Certora's CEO Mooly Sagiv, which of the following is generally accepted?}\label{sec:raceXcertora_q23}\\

\textbf{Solution: C \& D.}

\textbf{24. Based on the lecture on \href{https://www.youtube.com/watch?v=VGSsPIsbb6U}{"Auditing and Formal Verification: Better together"} by Certora's CEO Mooly Sagiv, the takeaways are}\label{sec:raceXcertora_q24}\\

\textbf{Solution: A, B \& C.}