\section{Security Pitfalls \& Best Practices 201 Quiz Solutions}

\textbf{1. The \texttt{InSecureumToken} contract strictly follow the specification of}\label{sec:exam5_q1}\\

\textbf{Solution: D.}

Since decimals was defined as \verb|uint| (same as \verb|uint256|) and not as \verb|uint8| as \verb|ERC20| or \verb|ERC777| standardized, it can't strictly be either of them.
And since this is clearly a fungible token contract, it can't be \verb|ERC721|.\\

\textbf{2. To avoid lock of funds, the following feature(s) MUST be implemented before contract deployment}\label{sec:exam5_q2}\\

\textbf{Solution: C.}

Contracts that accept Ether via \verb|payable| functions but without withdrawal mechanisms will lock up that Ether.
Remove \verb|payable| attribute or add a withdraw function.\\

\textbf{3. Which of the following assertion(s) is/are true (without affecting the security posture of the contract)?}\label{sec:exam5_q3}\\

\textbf{Solution: C \& D.}

\verb|buy()| cannot function without being \verb|payable|.\\
There's no reason the visibility of balances needs to be \verb|private|.\\
\verb|transfer()| can be \verb|external| since it's not called internally.\\
\verb|safeAdd()| can be \verb|public| since it is a \verb|pure| function.\\

\textbf{4. The total supply is limited by}\label{sec:exam5_q4}\\

\textbf{Solution: D.}

It would be B, but \verb|MAX_SUPPLY| isn't actually used anywhere in the code.\\

\textbf{5. The following issue(s) is/are present in the codebase}\label{sec:exam5_q5}\\

\textbf{Solution: B.}

It's impossible to get any Ether out of this contract, so no draining.\\
It divides \verb|desired_tokens| first and only then multiplies by the decimals, this causes any amount of tokens below 10 to result in 0 \verb|required_wei_sent|.\\
There are no divisions here that could allow a division by 0.\\

\pagebreak

\textbf{6. The following issue(s) is/are present in the codebase}\label{sec:exam5_q6}\\

\textbf{Solution: C.}

No requests made before/after a function call would be able to change the token price.\\
All of the functions are intended to be used by users, so no "access control" would be possible without excluding users.\\
A user can send all of their tokens to themselve, which will double their balance due to the pre-loaded variable reuse.\\

\textbf{7. The following issue(s) is/are present in the codebase}\label{sec:exam5_q7}\\

\textbf{Solution: D.}

No reentrancies are possible since no external calls are made.\\

\textbf{8. The following issue(s) is/are present in the codebase}\label{sec:exam5_q8}\\

\textbf{Solution: D.}

While it is indeed possible to exploit an overflow at the multiplication\linebreak (\verb|(desired_tokens / 10) * decimals|), it doesn't allow you to receive FREE tokens (although it makes them a bargain).\\