\section{Solidity 101 Quiz Solutions}

\textbf{1. User from EOA \texttt{A} calls Contract \texttt{C1} which makes an external call (\texttt{CALL} opcode) to Contract \texttt{C2}.
Which of the following is/are true?}\label{sec:exam2_q1}\\

\textbf{Solution: A \& C.}

Block and Transaction Properties:
\begin{itemize}
    \item\verb|msg.sender| (\verb|address|): sender of the message (current call)
    \item\verb|msg.value| (\verb|uint|): number of wei sent with the message
    \item\verb|tx.origin| (\verb|address|): sender of the transaction (full call chain)
\end{itemize}

The values of all members of \verb|msg|, including \verb|msg.sender| and \verb|msg.value| can change for every external function call.
This includes calls to library functions.\\

\textbf{2. Which of the following is/are true for \texttt{call}/\texttt{delegatecall}/\texttt{staticcall} primitives?}\label{sec:exam2_q2}\\

\textbf{Solution: A \& C.}

\verb|Call|/\verb|Delegatecall|/\verb|Staticcall|: In order to interface with contracts that do not adhere to the ABI, or to get more direct control over the encoding, the functions \verb|call|, \verb|delegatecall| and \verb|staticcall| are provided.
They all take a single bytes memory parameter and return the success condition (as a bool) and the returned data (bytes memory).\\

With \verb|delegatecall|, only the code of the given address is used but all other aspects (\verb|storage|, \verb|balance|, \verb|msg.sender|, etc\dots) are taken from the current contract.
The purpose of \verb|delegatecall| is to use library/logic code which is stored in callee contract but operate on the state of the caller contract.\\

With \verb|staticcall|, the execution will revert if the called function modifies the state in any way\\

\textbf{3. The gas left in the current transaction can be obtained with}\label{sec:exam2_q3}\\

\textbf{Solution: B.}

Block and Transaction Properties:

\begin{itemize}
    \item\verb|block.gaslimit| (\verb|uint|): current block \verb|gaslimit|
    \item\verb|tx.gasprice| (\verb|uint|): gas price of the transaction
    \item\verb|gasleft()| returns (\verb|uint256|): remaining gas
\end{itemize}

The function \verb|gasleft()| was previously known as \verb|msg.gas|, which was deprecated in version \verb|0.4.21| and removed in version \verb|0.5.0|.\\

\textbf{4. The default value of}\label{sec:exam2_q4}\\

\textbf{Solution: A, B, C \& D.}

A variable which is declared will have an initial default value whose byte-representation is all zeros.
The ``default values'' of variables are the typical ``zero-state'' of whatever the type is.\\

For example, the default value for a \verb|bool| is \verb|false|.
The default value for the \verb|uint| or \verb|int| types is \verb|0|.
For statically-sized arrays and \verb|bytes1| to \verb|bytes32|, each individual element will be initialized to the default value corresponding to its type.
For dynamically-sized arrays, \verb|bytes| and \verb|string|, the default value is an empty array or \verb|string|.
For the \verb|enum| type, the default value is its first member.\\

\textbf{5. Which of the following is/are true about events?}\label{sec:exam2_q5}\\

\textbf{Solution: A \& C.}

Events are an abstraction on top of the EVM's logging functionality.
Emitting events cause the arguments to be stored in the transaction's log (a special data structure in the blockchain).
These logs are associated with the address of the contract, are incorporated into the blockchain and stay there as long as a block is accessible.
The Log and its event data is not accessible from within contracts (not even from the contract that created them).
Applications can subscribe and listen to these events through the RPC interface of an Ethereum client.\\

Adding the attribute indexed for up to three parameters adds them to a special data structure known as ``topics'' instead of the data part of the log.
If you use arrays (including string and bytes) as indexed arguments, its Keccak-256 hash is stored as a topic instead, this is because a topic can only hold a single word (32 bytes).
All parameters without the indexed attribute are ABI-encoded into the data part of the log.
Topics allow you to search for events, for example when filtering a sequence of blocks for certain events.
You can also filter events by the address of the contract that emitted the event.\\

\textbf{6. Function \texttt{foo()} uses \texttt{block.number}. Which of the following is/are always true about \texttt{foo()}?}\label{sec:exam2_q6}\\

\textbf{Solution: D.}

Because it wouldn't always be true unless you know what other things \verb|foo()| uses too.\\

\pagebreak

\textbf{7. Solidity functions}\label{sec:exam2_q7}\\

\textbf{Solution: B, C \& D.}

Functions that are defined outside of contracts are called ``free functions'' and always have implicit internal visibility.
Their code is included in all contracts that call them, similar to internal library functions.\\

Function return variables are declared with the same syntax after the \verb|returns| keyword.
The names of return variables can be omitted.
Return variables can be used as any other local variable and they are initialized with their default value and have that value until they are (re-)assigned.\\

Function parameters are declared the same way as variables, and the name of unused parameters can be omitted.
Function parameters can be used as any other local variable and they can also be assigned to.\\

Variables and other items declared outside of a code block, for example functions, contracts, user-defined types, etc\dots, are visible even before they were declared.
This means you can use state variables before they are declared and call functions recursively.\\

\textbf{8. Conversions in \texttt{Solidity} have the following behavior}\label{sec:exam2_q8}\\

\textbf{Solution: B.}

An implicit type conversion is automatically applied by the compiler in some cases during assignments, when passing arguments to functions and when applying operators.
Implicit conversion between value-types is possible if it makes sense semantically and no information is lost.\\

However, if the compiler does not allow implicit conversion but you are confident a conversion will work, an explicit type conversion is sometimes possible.
This may result in unexpected behavior and allows you to bypass some security features of the compiler e.g. \verb|int| to \verb|uint|.
If an integer is explicitly converted to a smaller type, higher-order bits are cut off.
If an integer is explicitly converted to a larger type, it is padded on the left (i.e., at the higher order end).\\

\textbf{9. When Contract \texttt{A} attempts to make a \texttt{delegatecall} to Contract \texttt{B} but a prior transaction to Contract \texttt{B} has executed a \texttt{selfdestruct}}\label{sec:exam2_q9}\\

\textbf{Solution: C.}

The low-level functions \verb|call|, \verb|delegatecall| and \verb|staticcall| return \verb|true| as their first return value if the account called is non-existent, as part of the design of the EVM.
Account existence must be checked prior to calling if needed.\\

\pagebreak

\textbf{10. If \texttt{a = 1} then which of the following is/are true?}\label{sec:exam2_q10}\\

\textbf{Solution: A \& D.}

Operators Involving LValues (i.e. a variable or something that can be assigned to):\\

\verb|a += e| is equivalent to \verb|a = a + e|.
The operators \verb|-=|, \verb|*=|, \verb|/=|, \verb|%=|, \verb!|=!, \verb|&=| and \verb|^=| are defined accordingly.
\verb|a++| and \verb|a--| are equivalent to \verb|a += 1| and \verb|a -= 1| but the expression itself still has the previous value of \verb|a|.
In contrast, \verb|--a| and \verb|++a| have the same effect on \verb|a| but return the value after the change\\

\textbf{11. \texttt{transfer} and \texttt{send} primitives}\label{sec:exam2_q11}\\

\textbf{Solution: A, B \& D.}

The \verb|receive| function is executed on a call to the contract with empty \verb|calldata|.
This is the function that is executed on plain Ether transfers via \verb|.send()| or \verb|.transfer()|.
In the worst case, the receive function can only rely on 2300 gas being available (for example when \verb|send| or \verb|transfer| is used), leaving little room to perform other operations except basic logging.\\

\verb|.transfer(uint256 amount)| sends given \verb|amount| of wei to \verb|address|, reverts on failure, forwards 2300 gas stipend, not adjustable.\\

\verb|.send(uint256 amount)| sends given amount of wei to \verb|address|, returns false on failure, forwards 2300 gas stipend, not adjustable.\\

\textbf{12. A contract can receive Ether via}\label{sec:exam2_q12}\\

\textbf{Solution: A, B, C \& D.}

A contract without a \verb|receive| Ether function can receive Ether as a recipient of a \verb|coinbase| transaction (aka miner block reward) or as a destination of a \verb|selfdestruct|.\\

A contract cannot react to such Ether transfers and thus also cannot reject them.
This means that \verb|address(this).balance| can be higher than the sum of some manual accounting implemented in a contract (i.e. having a counter updated in the \verb|receive| Ether function).\\

A contract can have at most one \verb|receive| function, declared using\linebreak\verb|receive() external payable| without the \verb|function| keyword.
This is the function that is executed on plain Ether transfers via \verb|.send()| or \verb|.transfer()|.\\

Similarly, a contract can have at most one \verb|fallback| function, declared using either
\begin{lstlisting}[language=Solidity, style=solStyle]
fallback() external // it can also have the payable specifier
\end{lstlisting}

or

\begin{lstlisting}[language=Solidity, style=solStyle]
fallback (bytes calldata _input) external returns (bytes memory _output) // it can also have the payable specifier
\end{lstlisting}

both without the \verb|function| keyword.
This function must have \verb|external| visibility.
The \verb|fallback| function always receives data, but in order to also receive Ether it must be marked \verb|payable|.
In the worst case, if a \verb|payable fallback| function is also used in place of a \verb|receive| function, it can only rely on 2300 gas being available.\\

An \verb|Error(string)| exception (or an exception without data) is generated in the following situations: if your contract receives Ether via a \verb|public| function without \verb|payable| modifier (including the \verb|constructor| and the \verb|fallback| function).\\

\textbf{13. Structs in \texttt{Solidity}}\label{sec:exam2_q13}\\

\textbf{Solution: Correct is A, B \& C.}

Struct Types are custom defined types that can group several variables of same/different types together to create a custom data structure.
The \verb|struct| members are accessed using ``\verb|.|'' e.g.:

\begin{lstlisting}[style=defaultStyle]
struct s {address user; uint256 amount}
\end{lstlisting}

where \verb|s.user| and \verb|s.amount| access the struct members.\\

Mappings define key-value pairs and are declared using the syntax\linebreak\verb|mapping(_KeyType => _ValueType) _VariableName|.
The \verb|_KeyType| can be any built-in value type, \verb|bytes|, \verb|string|, or any \verb|contract| or \verb|enum| type.
Other user-defined or complex types, such as \verb|mappings|, \verb|structs| or array types are not allowed.
\verb|_ValueType| can be any type, including \verb|mappings|, arrays and \verb|structs|.\\

\textbf{14. The following is/are true about \texttt{ecrecover} primitive}\label{sec:exam2_q14}\\

\textbf{Solution: A \& C.}

Although internally it first recovers the public key from the signature, it actually returns the address derived from the public key.\\

\begin{lstlisting}[style=defaultStyle]
ecrecover(bytes32 hash, uint8 v, bytes32 r, bytes32 s) returns (address)
\end{lstlisting}

recovers the address associated with the public key from elliptic curve signature or return \verb|0| on error.
The function parameters correspond to ECDSA values of the signature: $r =$ first 32 bytes of signature, $s =$ second 32 bytes of signature, $v =$ final 1 byte of signature.
\verb|ecrecover| returns an \verb|address|, and not an \verb|address payable|.\\

If you use \verb|ecrecover|, be aware that a valid signature can be turned into a different valid signature without requiring knowledge of the corresponding private key.
This is usually not a problem unless you require signatures to be unique or use them to identify items.
\verb|OpenZeppelin| has a ECDSA helper library that you can use as a wrapper for \verb|ecrecover| without this issue.\\

\textbf{15. Which of the following is/are valid control structure(s) in \texttt{Solidity} (excluding \texttt{YUL})?}\label{sec:exam2_q15}\\

\textbf{Solution: A \& B.}

\verb|elif| is specific to \verb|Python|, and \verb|switch| has not been implemented in \verb|Solidity| yet.
\verb|Solidity| has \verb|if|, \verb|else|, \verb|while|, \verb|do|, \verb|for|, \verb|break|, \verb|continue|, \verb|return|, with the usual semantics known from \verb|C| or \verb|JavaScript|.\\

\textbf{16. \texttt{address} types}\label{sec:exam2_q16}\\

\textbf{Solution: B \& C.}

The \verb|address| type comes in two types:

\begin{enumerate}
    \item\verb|address|: Holds a 20 byte value (size of an Ethereum address)
    \item\verb|address payable|: Same as \verb|address|, but with the additional members \verb|transfer| and \verb|send|.
        \verb|address payable| is an address you can send Ether to, while a plain \verb|address| cannot be sent Ether.
\end{enumerate}

Members of Address Type:

\begin{itemize}
    \item\verb|address.balance (uint256)|: balance of the Address in Wei
    \item\verb|address.code (bytes memory)|: code at the Address (can be empty)
    \item\verb|address.call(bytes memory) returns (bool, bytes memory)|: issue low-level \verb|CALL| with the given payload, returns success condition and return data, forwards all available gas, adjustable
\end{itemize}

Implicit conversions from \verb|address payable| to \verb|address| are allowed, whereas conversions from \verb|address| to \verb|address payable| must be explicit via\linebreak\verb|payable(address)|.
Explicit conversions to and from \verb|address| are allowed for \verb|uint160|, integer literals, \verb|bytes20| and \verb|contract| types.\\

When attempting to add or subtract addresses, one gets

\begin{lstlisting}[style=defaultStyle]
TypeError: Operator - not compatible with types address and int_const 1. Arithmetic operations on addresses are not supported. Convert to integer first before using them.
\end{lstlisting}

\pagebreak

\textbf{17. If the previous block number was 1000 on Ethereum mainnet, which of the following is/are true?}\label{sec:exam2_q17}\\

\textbf{Solution: A, B \& C.}

Block number is the number of the block that is currently being mined, the next one.
Block number 1 was too long ago and its hash can no longer be accessed due to scaling reasons.
Mainnet ID Chain is 1.

Recall Block and Transaction Properties
\begin{itemize}
    \item\verb|blockhash(uint blockNumber) returns (bytes32)|: hash of the given block - only works for 256 most recent, excluding current, blocks
    \item\verb|block.chainid (uint)|: current chain id
    \item\verb|block.number (uint)|: current block number
    \item\verb|block.timestamp (uint)|: current block timestamp as seconds since unix epoch
\end{itemize}

\textbf{18. If we have an array then its data location can be}\label{sec:exam2_q18}\\

\textbf{Solution: A, B \& C.}

Every reference type has an additional annotation - the data location where it is stored.
There are three data locations: memory, storage and calldata.

\begin{itemize}
    \item whose lifetime is limited to an external function call
    \item whose lifetime is limited to the lifetime of a contract and the location where the state variables are stored
    \item which is a non-modifiable, non-persistent area where function arguments are stored and behaves mostly like memory.
        It is required for parameters of external functions but can also be used for other variables
\end{itemize}

\textbf{19. \texttt{delete varName;} has which of the following effects?}\label{sec:exam2_q19}\\

\textbf{Solution: A \& C.}

\verb|delete a| assigns the default value (whose byte-representation is all zeros) for the type to \verb|a|.\\

For integers it is equivalent to \verb|a = 0|.\\
For bools, it is equivalen to \verb|a = false|.\\
For structs, it assigns a struct with all members reset.\\
\verb|delete| has no effect on mappings.\\
If you delete a struct, it will reset all members that are not mappings and also recurse into the members unless they are mappings.\\

\pagebreak

\textbf{20. Which of the following is/are valid function specifier(s)?}\label{sec:exam2_q20}\\

\textbf{Solution: A, B \& C.}

Function Visibility Specifiers: \verb|public|, \verb|external|, \verb|internal| or \verb|private|.\\
Function Mutability Specifiers: \verb|pure| or \verb|view|.\\
In addition, there's also the \verb|payable| specifier, which allows a public function to receive Ether.\\

\textbf{State Variables} are the ones which can be declared as \verb|constant| or \verb|immutable|.\\

\textbf{21. Function visibility}\label{sec:exam2_q21}\\

\textbf{Solution: A.}

Default visibility is \verb|public| in current \verb|Solidity| versions.

\begin{itemize}
    \item Public functions are part of the contract interface and can be either called internally or via messages
    \item External functions are part of the contract interface, which means they can be called from other contracts and via transactions.
        An external function f cannot be called internally (i.e. \verb|f()| does not work, but \verb|this.f()| works)
    \item Internal functions can only be accessed internally from within the current contract or contracts deriving from it
    \item Private functions can only be accessed from the contract they are defined in and not even in derived contracts
\end{itemize}

\textbf{22. For error handling}\label{sec:exam2_q22}\\

\textbf{Solution: A, C \& D.}

\begin{itemize}
    \item\verb|assert(bool condition)|: causes a \verb|Panic| error and thus state change reversion if the condition is not met - to be used for internal errors
    \item\verb|require(bool condition)|: reverts if the condition is not met - to be used for errors in inputs or external components.
    \item\verb|require(bool condition, string memory message)|: reverts if the condition is not met - to be used for errors in inputs or external components.
        Also provides an error message.
    \item\verb|revert()|: abort execution and revert state changes
    \item\verb|revert(string memory reason)|: abort execution and revert state changes, providing an explanatory string
\end{itemize}

The \verb|assert| function creates an error of type \verb|Panic(uint256)|.
Assert should only be used to test for internal errors, and to check invariants.
Properly functioning code should never create a Panic, not even on invalid external input.\\

\textbf{23. Which of the following is/are true?}\label{sec:exam2_q23}\\

\textbf{Solution: C \& D.}

For constant variables, the value has to be a constant at compile time and it has to be assigned where the variable is declared.\\
The compiler does not reserve a storage slot for these variables, and every occurrence is replaced by the respective value.\\
Compared to regular state variables, the gas costs of constant and immutable variables are much lower.\\

\textbf{24. Integer overflows/underflows in \texttt{Solidity}}\label{sec:exam2_q24}\\

\textbf{Solution: B \& C.}

Integers in Solidity are restricted to a certain range.
For example, with \verb|uint32|, this is $0$ up to $2^{32} - 1$.
There are two modes in which arithmetic is performed on these types: The ``wrapping'' or ``unchecked'' mode and the ``checked'' mode.\\
By default, arithmetic is always ``checked'', which means that if the result of an operation falls outside the value range of the type, the call is reverted through a failing assertion.
You can switch to ``unchecked'' mode using \verb|unchecked|.
This was introduced in compiler version \verb|0.8.0|.\\

\textbf{25. Which of the following is true about mapping types in\linebreak\texttt{mapping(\_KeyType => \_ValueType)}?}\label{sec:exam2_q25}\\

\textbf{Solution: B \& C.}

The \verb|_KeyType| can be any built-in value type, \verb|bytes|, \verb|string|, or any \verb|contract| or \verb|enum| type.
Other user-defined or complex types, such as mappings, structs or array types are not allowed.\\

\verb|_ValueType| can be any type, including mappings, arrays and structs.
They can only have a data location of storage and thus are allowed for state variables, as storage reference types in functions, or as parameters for library functions.\\

You cannot iterate over mappings, i.e. you cannot enumerate their keys.
It is possible, though, to implement a data structure on top of them and iterate over that.\\

\pagebreak

\textbf{26. \texttt{receive()} and \texttt{fallback()} functions}\label{sec:exam2_q26}\\

\textbf{Solution: A, B \& D.}

A contract can have at most one \verb|fallback| function, declared using either

\begin{lstlisting}[language=Solidity, style=solStyle]
fallback() external // it can also have the payable specifier
\end{lstlisting}

or

\begin{lstlisting}[language=Solidity, style=solStyle]
fallback (bytes calldata _input) external returns (bytes memory _output) // it can also have the payable specifier
\end{lstlisting}

both without the \verb|function| keyword.
This function must have \verb|external| visibility.
The \verb|fallback| function always receives data, but in order to also receive Ether it must be marked \verb|payable|.
In the worst case, if a \verb|payable fallback| function is also used in place of a \verb|receive| function, it can only rely on 2300 gas being available.\\


\textbf{27. In \texttt{Solidity}, \texttt{selfdestruct(address)}}\label{sec:exam2_q27}\\

\textbf{Solution: B \& D.}

\verb|selfdestruct(address payable recipient)|: Destroy the current contract, sending its funds to the given Address and end execution.\\

\textbf{28. Which of the following is/are correct?}\label{sec:exam2_q28}\\

\textbf{Solution: A \& B.}

The version pragma indicates the specific \verb|Solidity| compiler version to be used for that source file and is used as follows: \verb|pragma solidity x.y.z;| where \verb|x.y.z| indicates the version of the compiler.\\

Using the version pragma does not change the version of the compiler.
It also does not enable or disable features of the compiler.
It just instructs the compiler to check whether its version matches the one required by the pragma.
If it does not match, the compiler issues an error.\\

A ``\verb|^|'' symbol prefixed to \verb|x.y.z| in the pragma indicates that the source file may be compiled only from versions starting with \verb|x.y.z| until \verb|x.(y+1).z|.
For e.g., \verb|pragma solidity ^0.8.3;| indicates that source file may be compiled with compiler version starting from \verb|0.8.3| until any \verb|0.8.z| but not \verb|0.9.z|.
This is known as a ``floating pragma''.\\

Complex pragmas are also possible using ``\verb|>|'', ``\verb|>=|'', ``\verb|<|'' and ``\verb|<=|'' symbols to combine multiple versions e.g. \verb|pragma solidity >=0.8.0 <0.8.3;|.\\

\pagebreak

\textbf{29. The impact of data location of reference types on assignments is}\label{sec:exam2_q29}\\

\textbf{Solution: D.}

In reality, they all do the opposite.\\

Assignments between storage and memory (or from calldata) always create an independent copy.\\
Assignments from memory to memory only create references.
This means that changes to one memory variable are also visible in all other memory variables that refer to the same data.\\
Assignments from storage to a local storage variable also only assign a reference.\\
All other assignments to storage always copy.
Examples for this case are assignments to state variables or to members of local variables of storage struct type, even if the local variable itself is just a reference.\\

\textbf{30. Which of the below are value types?}\label{sec:exam2_q30}\\

\textbf{Solution: A, B \& D.}

Value Types are the ones passed by value, i.e. they are always copied when they are used as function arguments or in assignments - Booleans, Integers, Fixed Point Numbers, Address, Contract, Fixed-size Byte Arrays (bytes1, bytes2, \dots, bytes32), Literals (Address, Rational, Integer, String, Unicode, Hexadecimal), Enums, Functions.\\

On the other hand, reference types can be modified through multiple different names.
Arrays (including Dynamically-sized bytes array bytes and string), Structs, Mappings.\\

\textbf{31. Arrays in \texttt{Solidity}}\label{sec:exam2_q31}\\

\textbf{Solution: A, B \& C.}

Arrays can have a compile-time fixed size, or they can have a dynamic size.
Indices are zero-based.\\

Array members:

\begin{itemize}
    \item\verb|length|: returns number of elements in array
    \item\verb|push()|: appends a zero-initialised element at the end of the array and returns a reference to the element
    \item\verb|push(x)|: appends a given element at the end of the array and returns nothing
    \item\verb|pop|: removes an element from the end of the array and implicitly calls delete on the removed element
\end{itemize}


\textbf{32. \texttt{Solidity} language is}\label{sec:exam2_q32}\\

\textbf{Solution: A, B, C \& D.}

Inline assembly support is given by the \verb|YUL| syntax.\\
\verb|Solidity| is statically typed, supports inheritance, libraries and complex user-defined types.
It is a fully-featured high-level language.
The syntax and OOP concepts are from \verb|C++|.