\section{Audit Findings 101 Quiz Solutions}

\textbf{1. Based on the comments and code shown in the \texttt{InSecureumDAO} snippet}\label{sec:exam7_q1}\\

\textbf{Solution: A, B \& C.}

While the payable \verb|openDAO()| function is protected by the correctly implemented \verb|onlyAdmin| modifier, it is always possible to force send Ether into a contract via \verb|selfdestruct()|.
The \verb|onlyWhenOpen()| modifier only checks for the contracts own balance which can be bypassed by doing that.
The payable \verb|join()| function indeed checks for the \verb|msg.value| to exactly match \verb|membershipFee|.\\

\textbf{2. Based on the comments and code shown in the \texttt{InSecureumDAO} snippet}\label{sec:exam7_q2}\\

\textbf{Solution: D.}

All state modifying functions, that can be accessed by users other than admin, are indeed correctly ``protected'' by the \verb|onlyWhenOpen| modifier, but that modifier is, as explained in the previous answer, not correctly implemented itself.
The only zero-address check is made during construction, but it's made against a state variable that will be in zero-state already (default value).
The zero-address check should instead be made against the \verb|_admin| variable which is to be set as the admin.
There are several functions that sound more than critical enough to have events (e.g. \verb|removeAllMembers|), but this contract isn't using events at all.\\

\textbf{3. Reentrancy protection only on \texttt{join()} (assume it's correctly specified) indicates that}\label{sec:exam7_q3}\\

\textbf{Solution: B.}

A simply sounds like nonesense.
Since it says that we should assume that reentrancy protection has been used correctly, and a reentrancy vulnerability requires making untrusted external calls, we can assume that it is at least likely, although not certain, that other functions do not.\\

\textbf{4. Access control on \texttt{msg.sender} for DAO membership is required in}\label{sec:exam7_q4}\\

\textbf{Solution: A \& B.}

 It wouldn't make much sense to pay a membership fee if you are allowed to create and cast votes without it.
 There's no clear reason to prevent non-members from accessing winning outcomes though, since they'd be publicly readable on the blockchain anyway.\\

\pagebreak

\textbf{5. A commit/reveal scheme (a cryptographic primitive that allows one to commit to a chosen value while keeping it hidden from others, with the ability to reveal the committed value later) is relevant for}\label{sec:exam7_q5}\\

\textbf{Solution: C.}

It's not possible to hide the \verb|msg.sender| using a simple commit/reveal scheme and it wouldn't make much sense to try, since there's no clear advantage from temporarily hiding your membership.
It also wouldn't make much sense to not disclose possible outcomes of a new vote, unless you want to make your members vote blindly on the options.
It does make sense to hide what you are voting for until voting closes, since this makes it impossible to calculate how many members voted for a specific option and how many fake/sibyl members you'd exactly need to create in order to manipulate the vote.\\

\textbf{6. Security concern(s) from missing input validation(s) is/are present in}\label{sec:exam7_q6}\\

\textbf{Solution: A \& D.}

The \verb|createVote()| function currently allows overwriting existing votes by specifying a previously used \verb|_voteId|.
It would probably be better to use an array instead of a mapping here and simply push new votes into it.
\verb|castVote()| has a modifier in place ensuring that a vote can only be cast on existing votes.
Since function body should be assumed as correctly implemented, we should also assume there are no security concerns in regards to validation either.
Without sanity/threshold checks when setting fees in \verb|setMembershipFee()|, admins could practically close the DAO off, preventing new members from joining, which could certainly be considered a security concern for the protocol.\\

\textbf{7. \texttt{removeAllMembers()} function}\label{sec:exam7_q7}\\

\textbf{Solution: A \& C.}

What the function actually does is only removing the admin as member from the DAO, he'd still stay the admin though.
Properly implementing this function would actually be rather difficult, since a simple delete on the mapping variable without specifying a key would not actually delete any of its values.
Assuming it would work as advertised by its name you can certainly say it would be a critical function that should emit an event, which it currently does not.\\

\textbf{8. \texttt{InSecureumDAO} will \textit{not} be susceptible to something like the 2016 ``DAO exploit''}\label{sec:exam7_q8}\\

\textbf{Solution: B.}

The 2016 ``DAO exploit'' was indeed a reentrancy issue caused by an external call within an Ether withdrawal function.
But simply inheriting from \verb|ReentrancyGuard.sol| will not prevent them since you actually have to apply the \verb|nonReentrant| modifier to relevant functions.
There have indeed been some efforts to prevent reentrancy issues with changes made in Ethereum and \verb|Solidity|, but none of them can be considered a ``fix''.\\

A recurrence of the 2016 ``DAO exploit'' is indeed still possible, although more unlikely since, thanks to all the attention it got, this anti-pattern is now widely known and rarely found anymore during audits.\\